
\documentclass{IOS-Book-Article}

\usepackage[utf8]{inputenc}
\usepackage{mathptmx}
\usepackage{soul}\setuldepth{article}


\usepackage{algpseudocode}

\usepackage{hyperref}
\usepackage{amsmath}
\usepackage{CJKutf8}
\usepackage{graphicx,pstricks}
\usepackage{ulem}
\usepackage{tikz}
\usepackage{algpseudocode}

\usepackage{moreverb}
\usepackage{subcaption}
\usepackage{epsfig}
\usepackage{multirow}
\usepackage[percent]{overpic}
\usepackage{textcomp}
\usepackage{subfiles}
\usepackage{enumitem}
\usepackage{tcolorbox}
\usepackage{algorithmicx}
\usepackage{ctable}
\usepackage{txfonts}
\usepackage{palatino}
\begin{comment}
\end{comment}
\usepackage{csquotes}
\renewcommand{\mkbegdispquote}[2]{\itshape}
\usepackage{textcomp}
\usepackage{url}
% fg settings
\newcommand{\setttsize}[1]{\def\ttsize{#1}}%
\newcommand{\mvmtmldir}[0]{/Users/psychomugs/projects/mvmt_ml_results/blossom_face/}
% ubicomp settings
\newcommand{\tablevspace}[0]{-.3cm}
\newcommand{\figwidth}[0]{0.9}
% \newcommand{\ms}[1]{\textcolor{red}{$<$MS: #1$>$}}
\newcommand{\ms}[1]{\textcolor{red}{#1}}
\newcommand\tab[1][1cm]{\hspace*{#1}}
\def\hb{\hbox to 11.5 cm{}}

\begin{document}
\begin{CJK*}{UTF8}{min}

\pagestyle{headings}
\def\thepage{}

\begin{frontmatter}              % The preamble begins here.


\title{
At Least, Be Human:\\
Humanizing the Robot as a\\
Medium for Communication}
% \title{
% At Least, Be Human: Humanizing the\\
% Robot as a Medium for Communication}

\markboth{}{August 2022\hb}

\author{\fnms{Michael} \snm{Suguitan}%
\thanks{E-mail:
mjs679@cornell.edu}}

\runningauthor{M. Suguitan}
\address{Cornell University}

\begin{abstract}
A central tenet of human-robot interaction research is to humanize social robots as collaborators and companions.
Literal approaches to humanizing, either through human-like behaviors or humanoid embodiments, pose technological and social challenges that have precluded adoption of robots in everyday contexts.
Even if convincingly humanized robots could be achieved, human-robot interaction may enable an escapism that diminishes our capacity for human-human interaction.
I propose to avoid these pitfalls by humanizing the robot as a medium for communication through accessibility.
Accessibility humanizes technology by makings inner workings visible and familiar to human users, promoting understanding of technological processes and imperfections.
Accessibility also enables broader demographics of lay users to become involved with robotics, enabling communication \textit{through} robots, from development processes (e.g. physical and behavioral design) to applications (e.g. telepresence).
I use the open-source Blossom social robot as an extended case study of this approach, and detail its technical implementations and research deployments.
The goal of this work is to present accessibility as a way to humanize robots while enabling robot-mediated communication for human-human interaction.
\end{abstract}

\begin{keyword}
human-robot interaction\sep robot-mediated communication\sep\\
accessibility\sep design\sep artificial intelligence\sep telepresence
\end{keyword}
\end{frontmatter}
\markboth{August 2022\hb}{August 2022\hb}
%!TEX root=../robophilosophy.tex

\section{Introduction}
Drawing from several definitions of social robots \cite{breazeal2004designing,dautenhahn1999bringing,duffy2003anthropomorphism,fong2003survey}, a prevailing notion is the objective of humanizing robots to facilitate social interactions.
Humanizing is often taken literally through two objectives: human-like intelligence or lifelike humanoid embodiments \cite{giger2019humanization,mays2021humanizing}.
However, these approaches pose large technological and social challenges.
Consumer social robots (e.g. Anki's Cozmo and Vector \cite{ankivector}, Jibo \cite{jibo}, Kuri \cite{kuri}) have had difficulty finding commercial success; this may be attributed to the high expectations for interaction -- often set by fiction \cite{komatsu2012expectations,sandoval2014human} -- that are difficult to attain.
Conversely, humanoid robots (e.g. Geminoids \cite{nishio2007geminoid}, Sophia \cite{hansonsophia}) and their imperfect approximation of human likeness may fail to meet our expectations for interaction, eliciting an unease that lands these robots in Mori's uncanny valley (Figure~\ref{fig:related.bukimi})~\cite{mori1970uncanny}.
The challenging expectations of literal humanization have limited acceptance of social robots in contexts beyond controlled research settings.
Even if sufficiently capable robots could be realized, human-robot interaction may yield scenarios that, like other technologies such as mobile computing and entertainment media, may negatively affect our capacity for human-human interaction.

In this work, I propose an alternative approach to humanizing robots through accessibility.
Accessibility can enable lay users without prior robotics experience to become familiar with typically esoteric aspects of robotics, particularly its physical and behavioral design.
In contrast to typical human-robot interaction scenarios, which involves users communicating \textit{with} a robot, accessibility reframes the social role of the robot as a medium \textit{through} which users communicate.
Example scenarios include personalization of the robot's design and an application of the robot as an embodied telepresence platform.
In this article, I use the open-source Blossom social robot as an extended case study and detail three phases of its development: design, movement, and telepresence.
The primary objectives of this work are to present accessibility as an approach to humanizing robots, while hopefully inspiring other applications of accessibility that enable robot-mediated communication for human-human interaction.

\begin{figure}[t]
    \centering
    \includegraphics[trim={13cm 1cm 14cm 7cm},clip,width=0.7\columnwidth]{figures/related.bukimi}
    \caption{Our proposed approach for humanizing the robot, referencing Mori's \textit{bukimi no tani}~(不気味の谷, ``the valley of eerieness,'' anglicized as ``the uncanny valley'') as a conceptual framework \cite{mori1970uncanny}. I equate ``humanizing'' with maximizing ``human affinity'' on the vertical axis of the graph (left green arrow), and journey out of the valley by making accessible (gray arrow) three phases of robot development: design, movement, and telepresence.}
    \label{fig:related.bukimi}
\end{figure}

\section{Preliminaries and Definitions, or \textit{What I Talk About When I Talk About...}}
In this section, I provide my definitions for key terms -- humanizing, medium, and communication -- and my rationale for pursuing accessibility as a humanizing element for robots.

\subsection{... Humanizing the Robot}
My approach to humanizing robots through accessibility begins with a reevaluation of Mori's \textit{bukimi no tani} (不気味の谷, ``the valley of eerieness''), anglicized as ``the uncanny valley.''
Originally in reference to the design and movement of prostheses, the phenomenon refers to slightly imperfect approximations of human features eliciting a sense of unease.
The graph's horizontal and vertical axes, originally \textit{ruijido}~(類似同, ``degree of similarity'') and \textit{shinwakan} (親和感, ``fellowship feeling''), are often anglicized as ``human likeness'' and ``affinity,'' respectively.
``Affinity'' itself is defined as ``a liking for or an attraction to something; a quality that makes people or things suited to each other'' \cite{affinity}; this notion is similar to familiarity\footnote{The first character of \textit{shinwakan}, 親, can be read as ``familiarity.'' Karl MacDorman interpreted \textit{shinwakan} as ``familiarity'' in his initial 2005 translation \cite{macdorman2005mortality}, then as ``affinity'' in his updated 2012 translation \cite{mori2012uncanny}.}. 
I argue that a ``human'' before ``affinity'' may have been lost in translation; reintroducing it yields ``human affinity,'' which I equate to the definition of ``humanizing.''
Thus, to humanize is to increase the feeling of human affinity and familiarity, which I interpret as maximizing the vertical position on the uncanny valley graph (Figure~\ref{fig:related.bukimi}, green arrow).

Due to the difficulty of literal humanization through human-familiar behaviors and embodiments, I propose to humanize robots through zoomorphism and accessibility.
The ``stuffed animal'' region of zoomorphic likeness lies at a local maxima before the fall into the valley.
Though it seems counter-intuitive to humanize the robot through non-humanoid zoomorphism, we often humanize animals and anthropomorphize personal belongings, e.g. musical instruments and vehicles \cite{jacob2012yeah}.
Zoomorphism can lower expectations for interaction while drawing upon our social history of animal companionship and familiarity \cite{darling2021new}.
Accessibility can promote familiarity by making the robot's inner workings visible and understandable to lay users without prior robotics experience.
This notion of humanizing through accessibility is echoed in post-digital aesthetics \cite{cascone2000aesthetics,cramer2015post,simbelis2018humanizing,thibault2018made}, which humanizes technology by making processes familiar to users and embracing imperfections, communicating the humanity of the technology's creators and users.
Post-digital's emphasis on processes draws attention to technological mediums, inviting interpretation of robots as mediums themselves.

\subsection{... Medium for Communication}
While robots in social contexts are typically interpreted as independent agents \textit{with} whom we communicate, we can also interpret them as technological mediums \textit{through} which we communicate.
Mediums can be any technology, artificial or natural, that enable communication, such as telephones which communicate auditory messages or photographs which communicate visual information; this relationship is formalized in Shannon's model of communication, wherein a message is encoded and decoded through a medium \cite{shannon1948communication}.
McLuhan, in declaring ``the medium is the message,'' argued that communicated messages (e.g. a particular phone conversation, a specific photograph of an event) are of secondary importance to mediums themselves and their ecological effects on environments of human communication \cite{mcluhan1964understanding}.
Hoorn has applied existing theories of computer-mediated communication in the context of robots as two distinct modes: human-robot communication (\textit{with} the robot) and robot-mediated communication (\textit{through} the robot) \cite{hoorn2020medium}.
Because of the aforementioned difficulty in realizing convincing autonomous agents, I focus on accessible robot-mediated communication between human users.

As McLuhan theorized, the effects of mediums both extend and amputate our capacity for communication.
The mobile smartphone extends our social connectivity but amputates our capacity for face-to-face communication; the photograph extends our visual communication but amputates our visual memory through the ``photo-taking-impairment effect'' \cite{henkel2014photo}.
Critics of social robots argue that robots pose similar risks for amputation, particularly in human-robot communication scenarios.
Turkle recounts users willing to replace their human partners with agreeable robots, becoming dependent on robots for matters as personal as health, or creating robotic replacements for the deceased \cite{turkle2017alone}.
Other scholars have argued that offloading human interactions to robots (e.g therapy, caretaking) may be dehumanizing for vulnerable populations \cite{calo2011ethical,williams2021misfit}.
These social amputations enable escapism from difficult human-human interactions towards the refuge of robots that neither tire nor disagree.

I believe that the robot-mediated mode offers opportunities for extending human communication, particularly through the unique affordance of physicality.
Though telepresence is the canonical application for robot-mediated communication, accessibility enables communication through other aspects of the robot, such as the design of its physical embodiment or the authoring of its behaviors.
Users can convey diverse interpretations through accessible robot design, challenging notions of how robots \textit{should} be by imagining how robots \textit{could} be.

\begin{figure*}[t]
    \centering
    \includegraphics[trim={0mm 25mm 0mm 25mm},clip,width=1.\textwidth]{figures/blossom_comm.jpeg}
    \caption{Blossom's journey out of the uncanny valley (left) and interpreting each phase (design, movement, telepresence) as forms of robot-mediated communication, as formalized through Shannon's model of communication (right) \cite{shannon1948communication}.}
    \label{fig:blossom_comm}
\end{figure*}

As an extended case study of our approach, I provide example deployments of the open-source Blossom social robot and detail three phases of its development: design, movement, and telepresence (Figure~\ref{fig:blossom_comm}).
We first move out of the valley through an open-source zoomorphic design;
accessible design enables lay users to communicate their notions of robot design through customization.
We next move further up through movement authoring that enables lay users to program robot behaviors;
accessible behavior programming enables lay users to communicate their interpretations of appropriate robot behaviors through high-level programming.
We finally move beyond the peak through embodied telepresence;
accessible telepresence enables lay users to communicate their physical presence at a distance.
Blossom's journey out of the uncanny valley represents accessibility as a way to humanize robots for robot-mediated communication.
% The following sections provide overviews of each phase, including technical implementations and research applications.

%!TEX root=../robophilosophy.tex

\section{Implementation on the Blossom Robot}
In this section, I discuss three phases of Blossom's development -- design, movement, and telepresence -- and how each phase humanizes through accessibility and enables robot-mediated communication.
I provide overviews of the technical implementations and research applications.

\subsection{Design}
\begin{figure*}[h]
    \centering
    % \includegraphics[trim={122mm 15mm 82mm 40mm},clip,width=1.\textwidth]{figures/robophil_design.jpeg}
    \includegraphics[trim={122mm 15mm 22mm 45mm},clip,width=1.\textwidth]{figures/design.anno.jpeg}

    \caption{Annotated portfolio \cite{gaver2012annotated} of Blossom and the aesthetic concepts that inspired its design.}
    \label{fig:design.anno}
\end{figure*}

Blossom's design is accessible through its open-source interior mechanism and user-customizable exterior (Figure~\ref{fig:design.anno}, center) \cite{suguitan2019blossom}.
The interior mechanism is constructed from laser cut wood and consists of a head platform suspended from a central tower component using rubber bands (Figure~\ref{fig:design.anno}, left).
Motors at the bottom of the tower actuate the head by reeling in strings attached to the head platform.
The tensile components achieve a large range of motion and passively smooth, lifelike movement, similar to the squash-and-stretch and follow-through principles of animation \cite{thomas1981illusion}.
The base robot features four degrees of freedom (roll, pitch, yaw, vertical translation); users can attach additional motors for appendages such as ears and arms.
The exterior is made of soft fabrics that are crafted by the user, inviting a broader range of users to be involved in robot building.
We have deployed Blossom in several contexts, ranging from demonstrations to in-depth robot-building workshops, where students (adolescents aged 10-13) worked in groups to build the interior mechanism, customize the exterior (Figure~\ref{fig:design.anno}, bottom right, top row), and choreograph movements to videos.
Other researchers have created Blossoms for their own applications (Figure~\ref{fig:design.anno}, bottom right, bottom row), including using Blossom as a ``canvas'' for exploring robot clothing \cite{friedman2021clothing}.

In the way that Norman frames artifacts as the medium through with designers indirectly communicate with their users \cite{norman2013design}, Blossom as an artifact communicates several aesthetic concepts that inspired its design.
Elements of post-digital design are present in the blending of analog and digital mediums (elastics and wood next to motors and microcomputers) and the inherent imperfection of its hand-crafted exterior.
Related to post-digital is \textit{kintsugi} (金継ぎ, ``golden repair''), a Japanese aesthetic that embraces imperfection by celebrating repair and our enduring relationships with objects \cite{richie2007tractate}.
The use of unconventional materials and resulting ``non-robotic'' embodiment appeals to tenets of critical design, a way to challenge preconceived notions of products and their roles in our lives \cite{dunnerabycritical}.
% Like W. Grey Walter's robotic tortoises, the progenitors of modern robots, 

% \ms{Any "clean" way to make this a discourse / aside-like section?}
Blossom's design also pays homage to the history of robots.
W. Grey Walter, a psychologist who created the first robotic tortoises that would become the progenitors of modern physical robots, found that even simple attraction or repulsion from light sources elicited lifelike behaviors, enough for Walter to bestow unique names to the respective robots: Elmer and Elsie \cite{walter1950imitation}.
Similarly, though Blossom is reducible to a simple assembly of motors and subroutines, the resulting expressiveness and lifelikeness suggests more than the sum of its components belie.
Drawing from Walter Benjamin's seminal \textit{The Work of Art in the Age of Mechanical Reproduction} \cite{benjamin1935work}, each Blossom's existence as a unique instantiation of the base robot imbues the artifact with an ``aura,'' the unique spatiotemporal quality lost with the commodification of mass-produced objects.
% Aura contrasts with the relationship of robots and industrialization, from both their use as machines for mechanical reproduction, and as mechanically reproduced machines themselves.

Beyond roboticists and designers, lay users can communicate their interpretations of robot aesthetics through Blossom's design (Figure \ref{fig:blossom_comm}, bottom right).
In the deployments, participants worked together to realize various robot aesthetics and related the designs to personal experiences, such as their favorite media and hobbies.
Users also used designs to project their ideal capabilities of the robot, such as functional appendages (e.g. arms, wings, tails) or locomotion.
We enable preliminary exploration of these capabilities by making accessible the authoring of movements, a form of message unique to the physical medium of robots.

\subsection{Movement}

\begin{figure}[h]
    \centering
    \includegraphics[trim={6cm 4cm 6cm 3cm},clip,width=.7\columnwidth]{figures/tele.system.jpeg}
    \caption{
        The movement authoring system.
        Users move the phone (left), and \texttt{DeviceOrientation} transmits the motion of the phone through \texttt{ngrok} and \texttt{socket.io} to the robot.
        The robot's back end inverse kinematics model calculates the motor positions required to match the phone's pose.
        For the telepresence application, \texttt{WebRTC} transmits a first-person video feed from a wide-angle camera embedded inside the robot's head to the phone interface.}
    \label{fig:mvmt.system}
\end{figure}
\begin{figure}[h!]
    \centering
    \includegraphics[width=0.9\columnwidth]{figures/face_mvmt.png}
    \caption{One of the behavior generation neural network models: a face→movement translation network. The movement variational autoencoder (VAE) learns compressed representations of the original movements (top left to right). An additional ResNet-based image encoder (bottom left) compresses images of facial expressions $x_f$ into the shared latent space $\{z_m,z_f\}$. Once the end-to-end network is trained, we can either generate new movements by sampling from $z_m$ and passing through the movement decoder (top left to right) or translate faces into movements by passing images through the face encoder and movement decoder (bottom left to right).}
    \label{fig:mvmt.vae}
\end{figure}

Blossom's movement is accessible through its motion-based smartphone interface (Figure~\ref{fig:mvmt.system}) \cite{suguitan2020moveae}.
Unlike traditional robot movement authoring systems that require high levels of expertise (e.g. motion planning software, manual robot operation techniques), the smartphone interface is familiar to users without prior robotics experience.
The interface transmits the phone's motion data to the robot's host computer, which then calculates the motor positions to match the robot's head orientation to the phone's orientation.
Using this interface, we crowdsourced movement samples from lay users by asking them to puppeteer the robot as if it were conveying a range of emotions (happiness, sadness, anger).
We used the crowdsourced movements as inputs for behavior generation models based on encoder-decoder neural networks, specifically variational autoencoders (VAE) \cite{kingma2013auto} (Figure \ref{fig:mvmt.vae}).
Once trained, the models could generate new movements, modify the emotive quality of existing movements \cite{suguitan2020moveae}, and translate affective inputs (e.g. facial expressions) into movements as emotive responses.

Blossom's accessible movement authoring system enables users to communicate their interpretations of robot behaviors (Figure \ref{fig:blossom_comm}, middle right).
Unlike closed-source robot systems, enabling users to ``teach'' new behaviors iteratively expands and personalizes its behavior library to stave off staleness.
The encoder-decoder architecture of the neural network models is analogous to the compression of Shannon's communication model.
By extending the movement control system with remote access through the internet, we enable real-time human-human communication through Blossom's telepresence capabilities.

\subsection{Telepresence}

Similar to movement, Blossom's telepresence is accessible through the ease of use of its interface \cite{youarenottherobot2021suguitan}.
Users can access the interface remotely through a mobile browser with no additional software.
A camera in Blossom's head streams a first-person video feed to the phone, enabling remote users to view the space as if they were embodying the robot (Figure~\ref{fig:mvmt.system}, left).
We have used the system in human evaluations (N=30), where users remotely controlled the robot to create emotive movements to expand the behavior dataset \cite{suguitan2022variable}.
We varied the viewpoint between either first-person (internal video \textit{from} the robot viewed on the phone interface) or third-person (external video \textit{of} the robot viewed on a separate desktop browser interface) perspectives.
We found large preferences for the third-person perspective, though the COVID-mandated social distancing restrictions prevented us from evaluating the system in a human-human interaction scenarios preferable for the first-person perspective.

Blossom's accessible telepresence is the most direct example of robot-mediated communication between human users (Figure \ref{fig:blossom_comm}, top right).
Unlike traditional button- or joystick-centric interfaces for screen-based telepresence robots that abstract users away from their own embodiment, the motion-based interface and Blossom's motion-centric embodiment enable a more direct communication of the remote user's physicality.
% I envision this system as a general embodied communication device with the unique affordance of physicality, granting users agency over their viewpoint and increasing their sense of presence in the remote location.
We envision this system as a general embodied communication device that transmits a user's physicality and increases their sense of presence in the remote location.

%!TEX root=../robophilosophy.tex
\section{Discussion, or \textit{A Portrait of the Robot as an Artistic Medium}}

% \subsection{Applications}
% The research we presented here represents just an initial exploration of the application of accessible robots.
% As most clearly illustrated in the telepresence case, robots may become a new class of embodied communication devices.
% Similar to how mobile phones enable a wide range of applications, from peer-to-peer communication to interfacing with medical equipment (e.g. glucose and heart monitors), 

% Telehealth, teletherapy

% \subsubsection{A Portrait of the Robot as an Artistic Medium}
% % Beyond utilitarian applications, 
% % Much research in robotics is devoted to the development of hardware and algorithms for utilitarian purposes, such as navigation or manipulation; recent advances in machine learning have dramatically accelerated progress in these fields.
% Much research in robotics is devoted to the development of hardware and algorithms for utilitarian purposes, such as navigation or manipulation; recent advances in machine learning have dramatically accelerated progress in these fields.
% Though robot artworks date back decades \cite{pagliarini2009development}, they have been relatively isolated as singular performances or constrained to the fictional works, perhaps owing to their technical inaccessibility.
% Drawing analogies to visual mediums, photography's roots in technological processes predisposed the medium to the objective documentation of reality.



% % Several roboticists 
% % , such as Stellarc's Reclining Stickman or the Charisma Lab's robot comedy performances \cite{swaminathan2021comedy}.
% \begin{itemize}
% \item What is the roboticist's most basic instrument -- the equivalent of the photographer's camera, the painter's brush, or the writer's pen -- through which she communicates? 
% \item What are the theories of robot art -- the equivalent of visual composition or linguistic grammar -- that outline the fundamental skills for robotic art?
% \item In the vein of the content of a medium itself being another medium, what medium contains robots? Trade shows and academic conferences are inaccessible for the general population.
% \end{itemize}


% \subsection{At Least, Be Human}

In contrast to the bulk of contemporary robotics research, which focuses on utilitarian applications that \textit{replace} humans, I interpret the robot as a medium that \textit{supports} human communication.
% The prior examples with Blossom are but exploratory case studies for robots as accessible mediums.
I envision the potential for robots to eventually become mediums for creative expression, capable of communicating artistic messages beyond the design and movement discussed here.
Though robotic artworks date back to their inception \cite{pagliarini2009development}, the technological inaccessibility of robots has rendered such works niche and sparse, with no unified theory or guidelines for practice.
Expanding the accessibility of robots to demographics beyond roboticists and human-robot interaction researchers will enable exploration of the robotic equivalents of tools (e.g. a writer's pen, a photographer's camera) and techniques (e.g. linguistic grammar, photographic composition) that yield theories and works of artistic communication through robots.

\section{Conclusion}
I presented accessibility as an approach for humanizing the robot as a medium for communication.
Drawing from the concept of the uncanny valley, I defined humanizing as an effort to maximize familiarity to human users.
Accessibility can humanize robots by making their inner workings visible and familiar to users, promoting understanding of the processes and imperfections of robotic technology.
Accessibility also enables users to communicate through phases of robot development, from its physical and behavioral design, as well as its application as an embodied communication platform.
I presented case studies for this approach using the Blossom robot and detailed how three phases of its development -- design, movement, and telepresence -- were made accessible and enabled users to communicate \textit{through} the medium of the robot.
I hope that this work inspires future roboticists and researchers to consider accessibility as an approach for humanizing robots as mediums for human-human communication.

\bibliographystyle{ieeetr}
\bibliography{master.bib}

\end{CJK*}
\end{document}
