%!TEX root=../robophilosophy.tex
\section{Discussion, or \textit{A Portrait of the Robot as an Artistic Medium}}

% \subsection{Applications}
% The research we presented here represents just an initial exploration of the application of accessible robots.
% As most clearly illustrated in the telepresence case, robots may become a new class of embodied communication devices.
% Similar to how mobile phones enable a wide range of applications, from peer-to-peer communication to interfacing with medical equipment (e.g. glucose and heart monitors), 

% Telehealth, teletherapy

% \subsubsection{A Portrait of the Robot as an Artistic Medium}
% % Beyond utilitarian applications, 
% % Much research in robotics is devoted to the development of hardware and algorithms for utilitarian purposes, such as navigation or manipulation; recent advances in machine learning have dramatically accelerated progress in these fields.
% Much research in robotics is devoted to the development of hardware and algorithms for utilitarian purposes, such as navigation or manipulation; recent advances in machine learning have dramatically accelerated progress in these fields.
% Though robot artworks date back decades \cite{pagliarini2009development}, they have been relatively isolated as singular performances or constrained to the fictional works, perhaps owing to their technical inaccessibility.
% Drawing analogies to visual mediums, photography's roots in technological processes predisposed the medium to the objective documentation of reality.



% % Several roboticists 
% % , such as Stellarc's Reclining Stickman or the Charisma Lab's robot comedy performances \cite{swaminathan2021comedy}.
% \begin{itemize}
% \item What is the roboticist's most basic instrument -- the equivalent of the photographer's camera, the painter's brush, or the writer's pen -- through which she communicates? 
% \item What are the theories of robot art -- the equivalent of visual composition or linguistic grammar -- that outline the fundamental skills for robotic art?
% \item In the vein of the content of a medium itself being another medium, what medium contains robots? Trade shows and academic conferences are inaccessible for the general population.
% \end{itemize}


% \subsection{At Least, Be Human}

In contrast to the bulk of contemporary robotics research, which focuses on utilitarian applications that \textit{replace} humans, I interpret the robot as a medium that \textit{supports} human communication.
% The prior examples with Blossom are but exploratory case studies for robots as accessible mediums.
I envision the potential for robots to eventually become mediums for creative expression, capable of communicating artistic messages beyond the design and movement discussed here.
Though robotic artworks date back to their inception \cite{pagliarini2009development}, the technological inaccessibility of robots has rendered such works niche and sparse, with no unified theory or guidelines for practice.
Expanding the accessibility of robots to demographics beyond roboticists and human-robot interaction researchers will enable exploration of the robotic equivalents of tools (e.g. a writer's pen, a photographer's camera) and techniques (e.g. linguistic grammar, photographic composition) that yield theories and works of artistic communication through robots.

\section{Conclusion}
I presented accessibility as an approach for humanizing the robot as a medium for communication.
Drawing from the concept of the uncanny valley, I defined humanizing as an effort to maximize familiarity to human users.
Accessibility can humanize robots by making their inner workings visible and familiar to users, promoting understanding of the processes and imperfections of robotic technology.
Accessibility also enables users to communicate through phases of robot development, from its physical and behavioral design, as well as its application as an embodied communication platform.
I presented case studies for this approach using the Blossom robot and detailed how three phases of its development -- design, movement, and telepresence -- were made accessible and enabled users to communicate \textit{through} the medium of the robot.
I hope that this work inspires future roboticists and researchers to consider accessibility as an approach for humanizing robots as mediums for human-human communication.