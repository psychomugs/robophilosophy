%!TEX root=../robophilosophy.tex

\section{Introduction}
Drawing from several definitions of social robots \cite{breazeal2004designing,dautenhahn1999bringing,duffy2003anthropomorphism,fong2003survey}, a prevailing notion is the objective of humanizing robots to facilitate social interactions.
Humanizing is often taken literally through two objectives: human-like intelligence or lifelike humanoid embodiments \cite{giger2019humanization,mays2021humanizing}.
However, these approaches pose large technological and social challenges.
Consumer social robots (e.g. Anki's Cozmo and Vector \cite{ankivector}, Jibo \cite{jibo}, Kuri \cite{kuri}) have had difficulty finding commercial success; this may be attributed to the high expectations for interaction -- often set by fiction \cite{komatsu2012expectations,sandoval2014human} -- that are difficult to attain.
Conversely, humanoid robots (e.g. Geminoids \cite{nishio2007geminoid}, Sophia \cite{hansonsophia}) and their imperfect approximation of human likeness may fail to meet our expectations for interaction, eliciting an unease that lands these robots in Mori's uncanny valley (Figure~\ref{fig:related.bukimi})~\cite{mori1970uncanny}.
The challenging expectations of literal humanization have limited acceptance of social robots in contexts beyond controlled research settings.
Even if sufficiently capable robots could be realized, human-robot interaction may yield scenarios that, like other technologies such as mobile computing and entertainment media, may negatively affect our capacity for human-human interaction.

In this work, I propose an alternative approach to humanizing robots through accessibility.
Accessibility can enable lay users without prior robotics experience to become familiar with typically esoteric aspects of robotics, particularly its physical and behavioral design.
In contrast to typical human-robot interaction scenarios, which involves users communicating \textit{with} a robot, accessibility reframes the social role of the robot as a medium \textit{through} which users communicate.
Example scenarios include personalization of the robot's design and an application of the robot as an embodied telepresence platform.
In this article, I use the open-source Blossom social robot as an extended case study and detail three phases of its development: design, movement, and telepresence.
The primary objectives of this work are to present accessibility as an approach to humanizing robots, while hopefully inspiring other applications of accessibility that enable robot-mediated communication for human-human interaction.

\begin{figure}[t]
    \centering
    \includegraphics[trim={13cm 1cm 14cm 7cm},clip,width=0.7\columnwidth]{figures/related.bukimi}
    \caption{Our proposed approach for humanizing the robot, referencing Mori's \textit{bukimi no tani}~(不気味の谷, ``the valley of eerieness,'' anglicized as ``the uncanny valley'') as a conceptual framework \cite{mori1970uncanny}. I equate ``humanizing'' with maximizing ``human affinity'' on the vertical axis of the graph (left green arrow), and journey out of the valley by making accessible (gray arrow) three phases of robot development: design, movement, and telepresence.}
    \label{fig:related.bukimi}
\end{figure}

\section{Preliminaries and Definitions, or \textit{What I Talk About When I Talk About...}}
In this section, I provide my definitions for key terms -- humanizing, medium, and communication -- and my rationale for pursuing accessibility as a humanizing element for robots.

\subsection{... Humanizing the Robot}
My approach to humanizing robots through accessibility begins with a reevaluation of Mori's \textit{bukimi no tani} (不気味の谷, ``the valley of eerieness''), anglicized as ``the uncanny valley.''
Originally in reference to the design and movement of prostheses, the phenomenon refers to slightly imperfect approximations of human features eliciting a sense of unease.
The graph's horizontal and vertical axes, originally \textit{ruijido}~(類似同, ``degree of similarity'') and \textit{shinwakan} (親和感, ``fellowship feeling''), are often anglicized as ``human likeness'' and ``affinity,'' respectively.
``Affinity'' itself is defined as ``a liking for or an attraction to something; a quality that makes people or things suited to each other'' \cite{affinity}; this notion is similar to familiarity\footnote{The first character of \textit{shinwakan}, 親, can be read as ``familiarity.'' Karl MacDorman interpreted \textit{shinwakan} as ``familiarity'' in his initial 2005 translation \cite{macdorman2005mortality}, then as ``affinity'' in his updated 2012 translation \cite{mori2012uncanny}.}. 
I argue that a ``human'' before ``affinity'' may have been lost in translation; reintroducing it yields ``human affinity,'' which I equate to the definition of ``humanizing.''
Thus, to humanize is to increase the feeling of human affinity and familiarity, which I interpret as maximizing the vertical position on the uncanny valley graph (Figure~\ref{fig:related.bukimi}, green arrow).

Due to the difficulty of literal humanization through human-familiar behaviors and embodiments, I propose to humanize robots through zoomorphism and accessibility.
The ``stuffed animal'' region of zoomorphic likeness lies at a local maxima before the fall into the valley.
Though it seems counter-intuitive to humanize the robot through non-humanoid zoomorphism, we often humanize animals and anthropomorphize personal belongings, e.g. musical instruments and vehicles \cite{jacob2012yeah}.
Zoomorphism can lower expectations for interaction while drawing upon our social history of animal companionship and familiarity \cite{darling2021new}.
Accessibility can promote familiarity by making the robot's inner workings visible and understandable to lay users without prior robotics experience.
This notion of humanizing through accessibility is echoed in post-digital aesthetics \cite{cascone2000aesthetics,cramer2015post,simbelis2018humanizing,thibault2018made}, which humanizes technology by making processes familiar to users and embracing imperfections, communicating the humanity of the technology's creators and users.
Post-digital's emphasis on processes draws attention to technological mediums, inviting interpretation of robots as mediums themselves.

\subsection{... Medium for Communication}
While robots in social contexts are typically interpreted as independent agents \textit{with} whom we communicate, we can also interpret them as technological mediums \textit{through} which we communicate.
Mediums can be any technology, artificial or natural, that enable communication, such as telephones which communicate auditory messages or photographs which communicate visual information; this relationship is formalized in Shannon's model of communication, wherein a message is encoded and decoded through a medium \cite{shannon1948communication}.
McLuhan, in declaring ``the medium is the message,'' argued that communicated messages (e.g. a particular phone conversation, a specific photograph of an event) are of secondary importance to mediums themselves and their ecological effects on environments of human communication \cite{mcluhan1964understanding}.
Hoorn has applied existing theories of computer-mediated communication in the context of robots as two distinct modes: human-robot communication (\textit{with} the robot) and robot-mediated communication (\textit{through} the robot) \cite{hoorn2020medium}.
Because of the aforementioned difficulty in realizing convincing autonomous agents, I focus on accessible robot-mediated communication between human users.

As McLuhan theorized, the effects of mediums both extend and amputate our capacity for communication.
The mobile smartphone extends our social connectivity but amputates our capacity for face-to-face communication; the photograph extends our visual communication but amputates our visual memory through the ``photo-taking-impairment effect'' \cite{henkel2014photo}.
Critics of social robots argue that robots pose similar risks for amputation, particularly in human-robot communication scenarios.
Turkle recounts users willing to replace their human partners with agreeable robots, becoming dependent on robots for matters as personal as health, or creating robotic replacements for the deceased \cite{turkle2017alone}.
Other scholars have argued that offloading human interactions to robots (e.g therapy, caretaking) may be dehumanizing for vulnerable populations \cite{calo2011ethical,williams2021misfit}.
These social amputations enable escapism from difficult human-human interactions towards the refuge of robots that neither tire nor disagree.

I believe that the robot-mediated mode offers opportunities for extending human communication, particularly through the unique affordance of physicality.
Though telepresence is the canonical application for robot-mediated communication, accessibility enables communication through other aspects of the robot, such as the design of its physical embodiment or the authoring of its behaviors.
Users can convey diverse interpretations through accessible robot design, challenging notions of how robots \textit{should} be by imagining how robots \textit{could} be.

\begin{figure*}[t]
    \centering
    \includegraphics[trim={0mm 25mm 0mm 25mm},clip,width=1.\textwidth]{figures/blossom_comm.jpeg}
    \caption{Blossom's journey out of the uncanny valley (left) and interpreting each phase (design, movement, telepresence) as forms of robot-mediated communication, as formalized through Shannon's model of communication (right) \cite{shannon1948communication}.}
    \label{fig:blossom_comm}
\end{figure*}

As an extended case study of our approach, I provide example deployments of the open-source Blossom social robot and detail three phases of its development: design, movement, and telepresence (Figure~\ref{fig:blossom_comm}).
We first move out of the valley through an open-source zoomorphic design;
accessible design enables lay users to communicate their notions of robot design through customization.
We next move further up through movement authoring that enables lay users to program robot behaviors;
accessible behavior programming enables lay users to communicate their interpretations of appropriate robot behaviors through high-level programming.
We finally move beyond the peak through embodied telepresence;
accessible telepresence enables lay users to communicate their physical presence at a distance.
Blossom's journey out of the uncanny valley represents accessibility as a way to humanize robots for robot-mediated communication.
% The following sections provide overviews of each phase, including technical implementations and research applications.
